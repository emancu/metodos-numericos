\documentclass[a4paper]{article}
\usepackage[spanish]{babel}
\title{Trabajo Práctico 1}

\usepackage[utf8]{inputenc}
\usepackage{caratula}
\usepackage{graphicx}
\usepackage{color}
\usepackage{listings}
\usepackage{float}


\setlength{\leftmargin}{2cm}
\setlength{\rightmargin}{2cm}
\setlength{\oddsidemargin}{-1cm}
\setlength{\evensidemargin}{-1cm}
\setlength{\topmargin}{-1cm}
\setlength{\textwidth}{18cm}
\setlength{\textheight}{25cm}

\usepackage{fancyhdr}
\pagestyle{fancy}
\fancyhf{}
\fancyhead [LO,LE]{\scriptsize Trabajo Práctico N$^{\circ}$1}
\fancyhead [RO,RE]{\scriptsize Mancuso, Mataloni, Tolchinsky}
\fancyfoot[CE,CO]{\thepage}
\renewcommand{\footrulewidth}{0.4pt}

\usepackage[pdftex, bookmarks=true, colorlinks, citecolor=black, linkcolor=black]{hyperref}
\usepackage{multirow}

\begin{document}

\materia{Métodos Numéricos}
\submateria{Primer Cuatrimestre de 2012}
\titulo{Aproximación de Ceros o Raíces de una función. }
\subtitulo{ Aproximación Numérica, Newton-Raphson, Bisección  }
\grupo{Trabajo Práctico N$^{\circ}$1}
\integrante{Mancuso Emiliano}{597/07}{emiliano.mancuso@gmail.com}
\integrante{Mataloni Alejandro}{706/07}{amataloni@gmail.com}
\integrante{Tolchinsky Lucas}{591/07}{lucas.tolchinsky@gmail.com}
\maketitle
\newpage

\addcontentsline{toc}{section}{Índice}
\tableofcontents

% Main project

\newpage

\begin{large}\textbf{Conjunto Independiente de Peso M\'aximo}\end{large} \\

\section{Resumen TODO}
El resumen, de no m ́as de 200 palabras, deber ́a explicar brevemente el trabajo realizado y las conclusiones de los autores de manera que pueda ser u ́til por s ́ı solo para dar una idea del contenido del trabajo. Las palabras clave, no m ́as de cuatro, deben ser t ́erminos t ́ecnicos que den una idea del contenido del trabajo para facilitar su bu ́squeda en una base de datos tem ́atica.

\section{Introducción Teórica TODO}
Contendr ́a una breve explicaci ́on de la base te ́orica que fundamenta los m ́etodos involu- crados en el trabajo, junto con los m ́etodos mismos. No deben incluirse demostraciones de propiedades ni teoremas, ejemplos innecesarios, ni definiciones elementales (como por ejemplo la de matriz sim ́etrica). En vez de definiciones b ́asicas es conveniente citar ejemplos de bibliograf ́ıa adecuada. Una cita vale m ́as que mil palabras.
\\

\subsection{Bisección}
También conocido como método de \textbf{Búsqueda binaria}, se basa en el teorema del valor intermedio. 
Requiere dividir varias veces a la mitad los subintervalos de [a,b] y, en cada paso, localizar la mitad que contenga a \textbf{p}, tal que \textbf{f(p) = 0}.
Si bien tiene una importante propiedad, que asegura la convergencia en una solución, esta convergencia puede ser muy lenta. Por eso, a menudo se utiliza para iniciar otros métodos mas eficientes, como por ejemplo \textbf{Newton-Raphson}.

\subsection{Newton-Raphson TODO}



\section{Desarrollo TODO}
Deben explicarse los m ́etodos num ́ericos que utilizaron y su aplicaci ́on al problema concreto involucrado en el trabajo pr ́actico. Se deben mencionar los pasos que si- guieron para implementar los algoritmos, las dificultades que fueron encontrando y la descripci ́on de c ́omo las fueron resolviendo. Explicar tambi ́en c ́omo fueron planteadas y realizadas las mediciones experimentales. Los ensayos fallidos, hip ́otesis y conjeturas equivocadas, experimentos y m ́etodos malogrados deben figurar en esta secci ́on, con una breve explicaci ́on de los motivos de estas fallas (en caso de ser conocidas).

\section{Resultados TODO}
Deben incluir los resultados de los experimentos, utilizando el formato m ́as adecuado para su presentaci ́on. Deber ́an especificar claramente a qu ́e experiencia corresponde cada resultado. No se incluir ́an aqu ́ı corridas de m ́aquina. Algo fundamental en su aprendizaje en la materia es la presentaci ́on de resultados de forma clara y concisa para el lector.

\section{Discusión TODO}
Se incluir ́a aqu ́ı un an ́alisis de los resultados obtenidos en la secci ́on anterior (se analizar ́a su validez, coherencia, etc.). Deben analizarse como m ́ınimo los  ́ıtems pedidos en el enunciado. No es aceptable decir que “los resultados fueron los esperados”, sin hacer clara referencia a la teor ́ıa a la cual se ajustan. Adem ́as, se deben mencionar los resul- tados interesantes y los casos “patol ́ogicos” encontrados.

\section{Conclusiones TODO}
Esta secci ́on debe contener las conclusiones generales del trabajo. Se deben mencionar las relaciones de la discusi ́on sobre las que se tiene certeza, junto con comentarios y observaciones generales aplicables a todo el proceso. Mencionar tambi ́en posibles extensiones a los m ́etodos, experimentos que hayan quedado pendientes, etc.

\newpage

\section{Apéndices}
\subsection{A - Enunciado}

En este trabajo práctico se deberá hacer un programa que, dadas las características del cuerpo (masa y coeficiente de rozamiento) y las condiciones iniciales (altura y velocidad de lanzamiento) calcule el momento en que se produce el primer impacto contra el suelo y la velocidad en ese instante. Asumir que la aceleración de la gravedad es $g = 9.81$ m/s$^2$.

Utilizando este programa:
\begin{enumerate}
  \item Analizar el efecto de la precisión numérica utilizada y las tolerancias adoptadas en la solución obtenida al calcular el primer impacto.
  
  \item Considerar que luego del primer impacto el cuerpo ``rebota'' en el suelo de manera no elástica, calcular la altura máxima del objeto luego del primer rebote y el momento del segundo impacto.

  \item ¿Cómo varía la energía mecánica ($E = g y +\dot{y}^2/2$) del cuerpo entre el lanzamiento y el segundo rebote? Estudiar el efecto de $\alpha$ (donde $\alpha = 0$ implica considerar el caso sin rozamiento), $f_r$ y la precisión numérica en esta variación.
\end{enumerate}

\subsection{B - Códigos Fuente}

\subsubsection{Stopping criteria}
\begin{verbatim}
bool stopping_criteria(double a, double b, double tolerance) {
  return fabs(a - b) < tolerance;
}
\end{verbatim}

\subsubsection{Bisección}
\begin{verbatim}
Result zero_bisection(Params *p, double (*fn)(Params *, double)) {
  int  iteracion = p->max_iterations;
  double a = p->a, b = p->b, m;
  Result res;

  while( --iteracion > 0 && !stopping_criteria(a,b, p->tolerance)) {
     res.zero = (b+a)/2;
    ( fn(p,a) * fn(p, res.zero) > 0 )? a = res.zero : b =  res.zero;
  }

  return res;
}
\end{verbatim}


\subsubsection{Newton-Raphson}
\begin{verbatim}
Result zero_newton(Params* p, double (*fn)(Params *, double), double (*deriv) (Params *, double)){
  double current = p->x, previous = 0.0;
  Result res;

  for(int i = p->max_iterations; i > 0 && !stopping_criteria(previous, current, p->tolerance); i--){
    previous = current;
    current = previous - (fn(p, previous)/deriv(p, previous));
  }

  res.speed = deriv(p,current);
  res.zero  = current;

  return res;
}
\end{verbatim}

\newpage

\section{Referencias TODO}
Es importante incluir referencias a libros, art ́ıculos y p ́aginas de Internet consultados durante el desarrollo del trabajo, haciendo referencia a estos materiales a lo largo del informe. Se deben citar tambi ́en las comunicaciones personales con otros grupos.

\begin{center}
	\includegraphics[scale=0.50]{Graficos/07-01.png}
\end{center}

\end{document}
