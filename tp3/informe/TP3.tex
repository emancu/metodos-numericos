\documentclass[a4paper]{article}
\usepackage[spanish]{babel}
\title{Trabajo Práctico 2}

\usepackage[utf8]{inputenc}
\usepackage{caratula}
\usepackage{graphicx}
\usepackage{color}
\usepackage{listings}
\usepackage{float}
\usepackage{amsmath}
\usepackage{amsfonts}
\usepackage{amssymb}
\usepackage{mathtools}


\setlength{\leftmargin}{2cm}
\setlength{\rightmargin}{2cm}
\setlength{\oddsidemargin}{-1cm}
\setlength{\evensidemargin}{-1cm}
\setlength{\topmargin}{-1cm}
\setlength{\textwidth}{18cm}
\setlength{\textheight}{25cm}

\usepackage{fancyhdr}
\pagestyle{fancy}
\fancyhf{}
\fancyhead [LO,LE]{\scriptsize Trabajo Práctico N$^{\circ}$3}
\fancyhead [RO,RE]{\scriptsize Mancuso, Mataloni, Tolchinsky}
\fancyfoot[CE,CO]{\thepage}
\renewcommand{\footrulewidth}{0.4pt}

\usepackage[pdftex, bookmarks=true, colorlinks, citecolor=black, linkcolor=black]{hyperref}
\usepackage{multirow}
\usepackage{multicol}

\begin{document}

\materia{Métodos Numéricos}
\submateria{Primer Cuatrimestre de 2012}
\titulo{Cuando pase el temblor...}
\subtitulo{Autovalores}
\grupo{Trabajo Práctico N$^{\circ}$3}

\integrante{Mancuso, Emiliano}{597/07}{emiliano.mancuso@gmail.com}
\integrante{Mataloni, Alejandro}{706/07}{amataloni@gmail.com}
\integrante{Tolchinsky, Lucas}{591/07}{lucas.tolchinsky@gmail.com}
\resumen{}

\maketitle

\newpage

\addcontentsline{toc}{section}{Índice}
\tableofcontents

% Main document

\newpage

\section{Introducción Teórica}

%Contendr ́a una breve explicaci ́on de la base te ́orica que fundamenta los m ́etodos involu- crados en el trabajo, junto con los m ́etodos mismos. No deben incluirse demostraciones de propiedades ni teoremas, ejemplos innecesarios, ni definiciones elementales (como por ejemplo la de matriz sim ́etrica). En vez de definiciones b ́asicas es conveniente citar ejemplos de bibliograf ́ıa adecuada. Una cita vale m ́as que mil palabras.

\section{Desarrollo}

En este trabajo práctico debemos calcular las \textit{frecuencias naturales} de un estacionamiento de varios pisos ante un hipotético terremoto, y como distribuir los automóviles para que se encuentre a salvo. Para esto, el trasfondo teórico propuesto por la cátedra toma en cuenta el peso de cada piso y el desplazamiento horizontal de cada uno respecto del suelo. Con estos datos armamos una ecuación por piso logrando un sistema de ecuaciones que en su forma matricial consiste en una \textbf{matriz de tres bandas}. Calculando los \textit{autovalores} de esta matriz es como obtenemos las \textit{frecuencias naturales} necesarias para el desarrollo del problema.

\subsection{Autovalores}
El primer paso en este trabajo consiste en calcular los \textit{autovalores} de una matriz y para lograrlo utilizamos el \textit{Algoritmo QR} que es un método iterativo basado en la \textit{factorización QR} de una matriz. Este algoritmo es en verdad bastante simple; siendo $A = Q_0 \times R_0$ la matriz a la cual queremos calcularle los \textit{autovalores}, formamos la matriz $A_1 = R_0 \times Q_0$ y repetimos lo mismo para calcular $A_2$. En definitiva, el algoritmo genera una serie de matrices de la forma

$$A_{i + 1} = R_i \times Q_i$$

la cual tiende a ser una matriz cuya diagonal principal contiene los \textit{autovalores} de \textit{A}. Cabe destacar que $A_i$ tiende a ser una \textbf{matriz diagonal} si \textit{A} es \textbf{simétrica} o a una \textbf{matriz triangular superior} si no lo es.\\

\subsection{Factorización QR}
Sabiendo esto el principal desafío del \textit{Algoritmo QR} es obtener la \textit{factorización QR} de la matriz para lo cual tuvimos en cuenta dos algoritmos conocidos: \textit{Householder} y \textit{Givens}.
En un principio consideramos utilizar el algoritmo de \textit{Householder} pues en una primera impresión parecía ser más sencillo de mantener, dado que en su iteración iésima pone en cero todos los valores debajo de la diagonal de la iésima columna, no es necesario preocuparse granularmente por cada operación que cancela un valor.\\
Sin embargo en un análisis posterior nos dimos cuenta que siendo que la matriz a factorizar es de tres bandas, el algoritmo de \textit{Givens} sólo necesita anular los valores de la banda que se encuentra por debajo de la diagonal y dado que en la clase teórica estudiamos que este algoritmo se comporta mejor con matrices ralas, consideramos que implementar \textit{Givens} era el acercamiento más apropiado para el problema.

\subsection{Rotaciones de Givens}

El método de \textit{Rotaciones de Givens} es utilizado para obtener la \textit{factorización QR} de una matriz que en nuestro caso es de tres bandas. Por esta razón los únicos valores que hacen falta eliminar son los de la banda por debajo de la diagonal principal de \textit{A}, que llamaremos $b_i$, como mostramos en la \texttt{figura 1}.

\begin{multicols}{2}
\begin{figure}[H]
$$
 \begin{pmatrix}
   a_1 & c_1 & \hdotsfor{4} \\
   b_2 & a_2 &  c_2 & \hdotsfor{3} \\
  \hdots & b_3 & a_3 & c_3 & \hdotsfor{2} \\
   & & \ddots \\
   \hdotsfor{2} & b_i & a_i & c_i & \hdots \\
   & & & & \ddots \\
   \hdotsfor{4} & b_n & a_n \\
 \end{pmatrix}
$$
 \caption{matriz de tres bandas. Los $\cdots$ representan ceros.}
\end{figure}

\begin{figure}[H]
$$
 \begin{pmatrix}
   a_1 & c_1 & c'_1 & \hdotsfor{3} \\
   \hdots & a_2 &  c_2 & c'_2 &\hdotsfor{2} \\
   \hdotsfor{2} & a_3 & c_3 & c'_3 & \hdots \\
   & & \ddots \\
   \hdotsfor{2} & a_{i - 1} & c_{i - 1} & c'_{i - 1} & \hdots \\
   \hdotsfor{3} & a_i & c_i & \hdots \\
   & & & & \ddots \\
   \hdotsfor{4} & b_n & a_n \\
 \end{pmatrix}
$$
 \caption{matriz $R_i$ de Givens. Los $\cdots$ representan ceros.}
\end{figure}
\end{multicols}

En cada iteración de \textit{Givens} el algoritmo transforma la matriz \textit{A} en una \textbf{matriz triangular superior} eliminando $b_i$ y agregando $c'_i$ por encima de la banda superior (\texttt{figura 2}). Para esto en cada iteración se arma una \textit{matriz de rotación} (que es igual a la identidad salvo por cuatro valores) y se la la multiplica por izquierda con $R_i$. La matriz \textit{Q} se obtiene multiplicando las \textit{matrices de rotación} anteriores.\\

A la hora de implementar el algoritmo decidimos optimizar el espacio y no guardar los ceros que contiene nuestra matriz, con lo cual la primer representación consistió de tres un \texttt{arrays}, uno por cada banda. Y dado que la banda inferior se convierte en ceros y aparece una nueva banda por encima de la $c_i$ no necesitamos de más estructura que esta para guardar los datos. El problema con esta propuesta fue que las multiplicaciones eran complicadas de programar y que no podíamos representar otro tipo de matriz. Descartamos esta estructura rápidamente y terminamos implementando las matrices como \texttt{arrays de doubles} aunque tuviéramos que guardar los ceros extra, además, dado que la matriz tiene una fila por cada piso y los edificios de estacionamiento difícilmente tengan más de 100 pisos, el espacio desperdiciado no es tan significativo.

\subsection{Algoritmo QR: criterios de parada}

El \textit{Algoritmo QR} es un algoritmo iterativo es decir que en cada iteración del procedimiento se mejora el resultado, pero cuándo terminar de iterar es algo que no está definido y depende de la situación. En nuestro caso sabemos que la matriz resultante tiende a ser \textbf{diagonal} o \textbf{triangular superior} y de cualquier manera la \textbf{triangular inferior} tiende a cero. Teniendo en cuenta esto, el criterio de parada que elegimos consiste en realizar la \textbf{suma del módulo} de los elementos de la \textbf{triangular inferior} y comparar este valor con la suma de la iteración anterior, si dicha suma es menor a un $\epsilon > 0$ (parámetro de nuestro programa), terminamos el algoritmo.\\
Lo cierto es que no hay garantía de que al acercarse la \textbf{triangular inferior} a cero los valores en la diagonal se parezcan a los \textit{autovalores} y es por esto que probar este criterio de parada y asegurarnos buenos resultados va a ser parte de nuestra experimentación.

\subsection{Heurísticas}


%Deben explicarse los m ́etodos num ́ericos que utilizaron y su aplicaci ́on al problema concreto involucrado en el trabajo pr ́actico. Se deben mencionar los pasos que si- guieron para implementar los algoritmos, las dificultades que fueron encontrando y la descripci ́on de c ́omo las fueron resolviendo. Explicar tambi ́en c ́omo fueron planteadas y realizadas las mediciones experimentales. Los ensayos fallidos, hip ́otesis y conjeturas equivocadas, experimentos y m ́etodos malogrados deben figurar en esta secci ́on, con una breve explicaci ́on de los motivos de estas fallas (en caso de ser conocidas).

\newpage

\section{Resultados}

%Deben incluir los resultados de los experimentos, utilizando el formato m ́as adecuado para su presentaci ́on. Deber ́an especificar claramente a qu ́e experiencia corresponde cada resultado. No se incluir ́an aqu ́ı corridas de m ́aquina. Algo fundamental en su aprendizaje en la materia es la presentaci ́on de resultados de forma clara y concisa para el lector.

\newpage

\section{Discusión}

%Se incluira aquı un analisis de los resultados obtenidos en la seccion anterior (se analizara su validez, coherencia, etc.). Deben analizarse como mınimo los ıtems pedidos en el enunciado. No es aceptable decir que “los resultados fueron los esperados”, sin hacer clara referencia a la teor ́ıa a la cual se ajustan. Adem ́as, se deben mencionar los resul- tados interesantes y los casos “patol ́ogicos” encontrados.

\newpage

\section{Conclusiones}
%Esta secci ́on debe contener las conclusiones generales del trabajo. Se deben mencionar las relaciones de la discusi ́on sobre las que se tiene certeza, junto con comentarios y observaciones generales aplicables a todo el proceso. Mencionar tambi ́en posibles extensiones a los m ́etodos, experimentos que hayan quedado pendientes, etc.

\newpage

\section{Apéndices}
\subsection{A - Enunciado}

El trabajo pr\'actico consiste en evaluar la resistencia s\'\i smica de un
edificio de varios pisos que funciona como estacionamiento, proponiendo un plan de
reubicaci\'on de los veh\'iculos lo m\'as eficiente posible.\\

\textbf{El modelo}\\
Consideremos un edificio de $n$ pisos como el de la Figura~1. Un modelo sencillo
para estudiar el efecto de un terremoto sobre el edificio consiste en
considerar cada piso $i=1,\dots,n$ como un bloque de masa $m_i$, unido a los
pisos adyacentes por medio de un conector el\'astico cuya acci\'on se parece
a la de un resorte. Para $i=0,\dots,n-1$, la uni\'on entre los pisos $i$ e
$i+1$ suministra una fuerza de restituci\'on
\begin{displaymath}
F_i\ =\ k_i (x_{i+1}-x_i),
\end{displaymath}
donde $x_i(t)\colon \mathbb{R}_+ \rightarrow \mathbb{R}$ representa el desplazamiento
horizontal del $i$-\'esimo piso en cada instante con respecto al suelo (asumimos que $i=0$ corresponde
al suelo y que $x_0=0$), y los $k_i\in\mathbb{R}_+$ representan los coeficientes
de rigidez. Aplicando la segunda ley de Newton del movimiento %$F = ma$ 
a cada secci\'on del edificio ($m_i\, a_i = F_i-F_{i-1}$, con $a_i$ la aceleraci\'on, que escribiremos como la derivada segunda de $x_i$), 
obtenemos el siguiente sistema de ecuaciones diferenciales ordinarias:
\begin{eqnarray*}
m_1 \ddot{x}_1 & = & -k_0 x_1 + k_1 (x_2-x_1) \nonumber \\
m_2 \ddot{x}_2 & = & -k_1 (x_2-x_1) + k_2 (x_3-x_2) \nonumber \\
m_3 \ddot{x}_3 & = & -k_2 (x_3-x_2) + k_3 (x_4-x_3) \nonumber \\
\vdots &  & \vdots \nonumber \\
m_n \ddot{x}_n & = & -k_{n-1} (x_n-x_{n-1}) \nonumber \\
\end{eqnarray*}
Escrito en forma matricial, este
sistema toma la forma $M\ddot{\mathbf{x}} = K\mathbf{x}$, 
donde $M\in\mathbb{R}^{n\times n}$ es una matriz
diagonal con las masas de los pisos y $K\in\mathbb{R}^{n\times n}$ es una matriz
tridiagonal con los coeficientes de rigidez adecuados. Como $m_i>0$ para
$i=1,\dots,n$, entonces $M$ tiene inversa y el sistema se puede reescribir
como $\ddot{\mathbf{x}} = (M^{-1} K) \mathbf{x} = A\mathbf{x}$, donde $A = M^{-1} K$ tiene autovalores negativos.

Sean $\lambda_1,\dots,\lambda_n$ los autovalores de $A$. Los valores
$\omega_i=\sqrt{-\lambda_i}$, para $i=1,\dots,n$, representan las frecuencias
naturales del sistema, e indican la estabilidad del edificio durante un
terremoto. Si la frecuencia del sismo es muy pr\'oxima a alguna de estas
frecuencias, hay riesgo de que el edificio entre en resonancia y colapse.\\

\textbf{El problema}

Nos encontramos en el 
estacionamiento de una importante concesionaria de autom\' oviles de una reconocida marca,
% dep\'osito de lavarropas de una conocida casa de electrodom\'esticos, 
y se avecina un terremoto sobre nuestra ciudad. 
Contamos con informaci\'on fidedigna provista por nuestro
informante en el Departamento de Geolog\'\i a de la FCEyN 
de que la frecuencia del terremoto ser\'a $\omega = 3\ \hbox{Hz}
= 3 \frac{1}{\hbox{seg}}$.

Para realizar c\'alculos simplificados podemos asumir que todos los autos 
se pueden agrupar en 2 categor\'ias: livianos, de masa $m_l$, y pesados, 
de masa $m_p > m_l$.
Adem\'as, $m_0$ es la masa propia del edificio correspondiente a cada piso.
De esta forma, si el piso $i$ tiene $l_i$ veh\'\i culos livianos y 
$p_i$ veh\'\i culos pesados, entonces su masa es $m_i = m_0 + l_i m_l + p_i m_p$. 
El problema que debemos resolver -y r\'apidamente- consiste en determinar
cu\'antos autos livianos y pesados debemos quitar de cada piso 
(reubic\'andolos en otros pisos) para que ninguna de las frecuencias 
naturales del edificio se encuentre a menos del 10\% de la frecuencia 
$\omega$ del terremoto.
La soluci\'on \'optima del problema es aquella que permite evitar que el
edificio colapse, reubicando la menor cantidad posible de autom\'oviles.\\

\textbf{El enunciado}

El trabajo pr\'actico consiste en implementar un programa que permita 
resolver este problema. La soluci\'on propuesta debe indicar cu\'antos 
autos livianos y cu\'antos pesados quitar de cada piso, y a qu\'e pisos 
se deben llevarlos. 

Deben proponerse (por lo menos) dos m\'etodos (es v\'alido que sean 
heur\'\i sticos) para obtener el plan de reubicaci\'on. El informe 
deber\'a contener los resultados de los experimentos realizados para
compararlos y evaluar cu\'al es mejor. 

El programa debe incluir una implementaci\'on de
alg\'un algoritmo para calcular los autovalores de una matriz cuadrada, que
deber\'a ser utilizado durante el proceso de decisi\'on. Sugerimos implementar
el algoritmo QR para el c\'alculo de autovalores. El programa debe tomar los
datos desde un archivo de texto con el siguiente formato:
\begin{eqnarray}
 & & n\ m_0\ m_l\ m_p \nonumber \\
 & & k_0\ k_1\ \dots\ k_{n-1} \nonumber \\
 & & l_1\ l_2\ \dots\ l_n \nonumber \\
 & & p_1\ p_2\ \dots\ p_n \nonumber
\end{eqnarray}
Se debe retornar la soluci\'on propuesta con este mismo formato.

El programa que obtenga la mejor redistribuci\'on de autom\'oviles se har\'a
acreedor a la medalla \emph{M\'etodos Num\'ericos 2012} al Mejor Redistribuidor Vehicular.\\

\vskip 15pt

\hrule

\vskip 11pt

{\bf \underline{Entrega Final}}

\begin{description}
  \setlength{\itemsep}{0pt}
  \setlength{\parskip}{0pt}
  \setlength{\parsep}{0pt}
 \item[Formato Electr\'onico:] jueves 28 de junio de 2012, hasta las 23:59 hs, a la direcci\'on: 

  {\emph{metnum.lab@gmail.com}}
%  \item[Formato f\'isico y experimentación en clase:] 13 de abril de 2012, de 17 a 21 hs.
 \item[Formato f\'isico:] viernes 29 de junio de 2012, de 17 a 19 hs.
 \item[Competencia entre grupos:] viernes 29 de junio de 2012, 19 hs.
 \item[Entrega de premios:] viernes 29 de junio de 2012, 20:30 hs.
\end{description}


\newpage

%\section{Referencias TODO}
%Es importante incluir referencias a libros, art ́ıculos y p ́aginas de Internet consultados durante el desarrollo del trabajo, haciendo referencia a estos materiales a lo largo del informe. Se deben citar tambi ́en las comunicaciones personales con otros grupos.


\end{document}
