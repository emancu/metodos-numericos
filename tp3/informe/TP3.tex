\documentclass[a4paper]{article}
\usepackage[spanish]{babel}
\title{Trabajo Práctico 2}

\usepackage[utf8]{inputenc}
\usepackage{caratula}
\usepackage{graphicx}
\usepackage{color}
\usepackage{listings}
\usepackage{float}
\usepackage{amsmath}
\usepackage{amsfonts}
\usepackage{amssymb}
\usepackage{mathtools}


\setlength{\leftmargin}{2cm}
\setlength{\rightmargin}{2cm}
\setlength{\oddsidemargin}{-1cm}
\setlength{\evensidemargin}{-1cm}
\setlength{\topmargin}{-1cm}
\setlength{\textwidth}{18cm}
\setlength{\textheight}{25cm}

\usepackage{fancyhdr}
\pagestyle{fancy}
\fancyhf{}
\fancyhead [LO,LE]{\scriptsize Trabajo Práctico N$^{\circ}$2}
\fancyhead [RO,RE]{\scriptsize Mancuso, Mataloni, Tolchinsky}
\fancyfoot[CE,CO]{\thepage}
\renewcommand{\footrulewidth}{0.4pt}

\usepackage[pdftex, bookmarks=true, colorlinks, citecolor=black, linkcolor=black]{hyperref}
\usepackage{multirow}
\usepackage{multicol}

\begin{document}

\materia{Métodos Numéricos}
\submateria{Primer Cuatrimestre de 2012}
\titulo{Filtrado de imágenes resueltas con ecuaciones lineales. }
\subtitulo{ Ecuaciones Lineales, Filtro de imagen }
\grupo{Trabajo Práctico N$^{\circ}$3}

\integrante{Mancuso, Emiliano}{597/07}{emiliano.mancuso@gmail.com}
\integrante{Mataloni, Alejandro}{706/07}{amataloni@gmail.com}
\integrante{Tolchinsky, Lucas}{591/07}{lucas.tolchinsky@gmail.com}
\resumen{}

\maketitle

\newpage

\addcontentsline{toc}{section}{Índice}
\tableofcontents

% Main document

\newpage

\section{Introducción Teórica}

%Contendr ́a una breve explicaci ́on de la base te ́orica que fundamenta los m ́etodos involu- crados en el trabajo, junto con los m ́etodos mismos. No deben incluirse demostraciones de propiedades ni teoremas, ejemplos innecesarios, ni definiciones elementales (como por ejemplo la de matriz sim ́etrica). En vez de definiciones b ́asicas es conveniente citar ejemplos de bibliograf ́ıa adecuada. Una cita vale m ́as que mil palabras.

\section{Desarrollo}

%Deben explicarse los m ́etodos num ́ericos que utilizaron y su aplicaci ́on al problema concreto involucrado en el trabajo pr ́actico. Se deben mencionar los pasos que si- guieron para implementar los algoritmos, las dificultades que fueron encontrando y la descripci ́on de c ́omo las fueron resolviendo. Explicar tambi ́en c ́omo fueron planteadas y realizadas las mediciones experimentales. Los ensayos fallidos, hip ́otesis y conjeturas equivocadas, experimentos y m ́etodos malogrados deben figurar en esta secci ́on, con una breve explicaci ́on de los motivos de estas fallas (en caso de ser conocidas).

\newpage

\section{Discusión y Resultados}

%Deben incluir los resultados de los experimentos, utilizando el formato m ́as adecuado para su presentaci ́on. Deber ́an especificar claramente a qu ́e experiencia corresponde cada resultado. No se incluir ́an aqu ́ı corridas de m ́aquina. Algo fundamental en su aprendizaje en la materia es la presentaci ́on de resultados de forma clara y concisa para el lector.

%Se incluira aquı un analisis de los resultados obtenidos en la seccion anterior (se analizara su validez, coherencia, etc.). Deben analizarse como mınimo los ıtems pedidos en el enunciado. No es aceptable decir que “los resultados fueron los esperados”, sin hacer clara referencia a la teor ́ıa a la cual se ajustan. Adem ́as, se deben mencionar los resul- tados interesantes y los casos “patol ́ogicos” encontrados.

\newpage

\section{Conclusiones}
%Esta secci ́on debe contener las conclusiones generales del trabajo. Se deben mencionar las relaciones de la discusi ́on sobre las que se tiene certeza, junto con comentarios y observaciones generales aplicables a todo el proceso. Mencionar tambi ́en posibles extensiones a los m ́etodos, experimentos que hayan quedado pendientes, etc.

\newpage

\section{Apéndices}
\subsection{A - Enunciado}
\newpage

%\section{Referencias TODO}
%Es importante incluir referencias a libros, art ́ıculos y p ́aginas de Internet consultados durante el desarrollo del trabajo, haciendo referencia a estos materiales a lo largo del informe. Se deben citar tambi ́en las comunicaciones personales con otros grupos.


\end{document}
